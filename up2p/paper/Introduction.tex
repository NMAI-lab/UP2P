\section{Introduction}
Wikis were first created by Ward Cunningham as an easy way to edit and maintain web pages directly through the browser, and have become the embodiment of collaborative work on Internet.

In Wikipedia, the coordinated efforts of the web community were leveraged to create a mammoth encyclopedia, which currently contains several million articles in dozens of languages. The usefulness of Wikipedia is obvious from its massive use as an information source, and its factual accuracy has been found to be comparable with that of a traditional encyclopedia \cite{NatureWikipedia}. 

However, critics of Wikipedia disagree with this assessment. Their argument is that since anybody, no matter their qualifications, can edit an article, then those editors are likely to introduce errors and misinformation in Wikipedia. In other words, the problem is that all editors cannot be \emph{trusted}. Traditional encyclopedias, and some recent alternative internet encyclopedias, such as Citizendium \footnote{http://www.citizendium.org/} and ScholarPedia\footnote{http://www.scholarpedia.org/}, address this problem by having only trusted experts edit their content.

The Wikipedia answer would be that the solution to the problem lies precisely in its cause: since anybody can edit articles, factual errors or vandalism are quickly identified and fixed. The underlying rationale is that the editors as a group ultimately reach a consensus, either naturally or following an arbitration process, and that this consensus can be trusted. 

However, a consensus can't always be reached. In Wikipedia, aticles on controversial topics often need to be protected from repeated editing by opposing factions (``edit wars"). Although this choice to accomodate a single viewpoint in Wikipedia may be defendable, this raises a more general question: should collaborative editing always lead to a single, authoritative version of each document? 

To the best of our knowledge, all Wiki systems promote this collaboration model. We explore here a different model, inspired by the principles of Peer-to-peer (P2P) file-sharing: users may maintain different versions of a document, by copying and editing their preferred versions, among those available. 

This collaboration model would solve some of the problems of Wikipedia, described above. Another scenario where this collaboration model to some extent already exists, is the case of learning material, in school or university. Instructors often prepare their teaching material by reusing and adapting material from various sources, and in turn make their own material available to others. Their work can be described as collaborative, without leading to the development of a single, authoritative version. In this sense, we argue that a Wiki system following the same collaboration principles would be a very useful tool, for this scenario and others.

In this paper, we describe a Wiki system built on top of a peer-to-peer (P2P) network, and which supports this collaboration model. Different versions of any article may exist in the network, and user queries return all the relevant versions, from which users must then choose their preferred version to download. The trust problem raised earlier remains relevant: instead of the system implicitly deciding which version of the document should be adopted (the latest version in the case of traditional Wikis), user may establish individual trust relationships, and use these as a basis to choose contributions from the available versions. In addition, this model generates a non-trivial versioning process. In the traditional Wiki collaboration model, there is a linear sequence of versions; in this case there is a tree, or even a lattice of versions, that is important to present to the user. 

Our work makes the following contributions. First, we demonstrate the application of P2P file-sharing principles to collaborative edition by means of a wiki system. Secondly, we formally define the semantics of wikilinks and of ancestry relationships as two types of binary relations between documents. We then use these semantics to define complex queries such as the transitive closure of an ancestry relation, and provide query answering algorithms that were implemented in our system. Finally, we propose several trust metrics and their implementation in a P2P file-sharing network.

The rest of this paper is organized as follows. After a brief survey of related work, we present the main functionality of P2Pedia, and its conceptual  design using the principles of file-sharing, in section \ref{sec:proposal}. In section \ref{sec:implementation} we describe the implementation of our system, and finally in section \ref{sec:evaluation} we propose some scenarios for using our system, before giving a brief evaluation of its technical characteristics.